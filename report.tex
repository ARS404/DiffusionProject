\documentclass[a4paper]{article}

\RequirePackage[T1,T2A]{fontenc}
\RequirePackage[utf8]{inputenc}
\RequirePackage[russian]{babel}

\RequirePackage[left=2cm,right=2cm, top=2cm,bottom=2cm,bindingoffset=0cm]{geometry}

\RequirePackage{amsmath}
\RequirePackage{amssymb}
\RequirePackage{amsfonts}
\RequirePackage{amsthm}

\RequirePackage{graphicx}
\RequirePackage{fancyhdr}
\RequirePackage{mathtools}
\RequirePackage{wrapfig}

\usepackage[colorlinks,citecolor=black,linkcolor=black,bookmarks=false,hypertexnames=true, urlcolor=blue]{hyperref}

\title{Отчёт по проекту}
\author{Зайцев Фёдр, Ознобихин Арсений}
\begin{document}
    \maketitle{}
    \tableofcontents
    \newpage

    \section{Постановка задачи}
    \subsection{Задача}
    Реализация 4 солверов ОДУ, порожденного диффузионной моделью (Euler, DDIM, EDM, DPM),
    и сравнение их качества.

    \subsection{Цели}
    \begin{itemize}
        \item Реализация солверов.
        \item Реализация необходимого для проведения экспериментов кода.
        \item Проведение экспериментов.
        \item Сравнение результатов
    \end{itemize}


    \section{Техническое описание экспериментов}
    Все эксперименты проводились на датасете CIFAR10, в качестве диффузионной модели была выбрана
    EDM. Каждый из солверов несколько раз запускался в с 50 и 200 степами и генерировал по 3000
    изображений. Далее проводился подсчёт FID и визуальное сравнение результатов.
    \par Более подробное описание процесса запуска экспериментов можно увидеть в
    \href{https://github.com/ARS404/DiffusionProject}{README}


    \section{Результаты экспериментов}
    \subsection{Описание результатов}
    Было проведено несколько экспериментов, с различным количеством шагов солверов:
    \begin{table}
        \begin{tabular}{ l | c | c | c | c }
            \textbf{Solver} & Euler & DDIM  & EDM   & DPM \\ \hline
            \textbf{FID}    & 11{,}55 & 354{,}51     & 10{,}21   & -
        \end{tabular}\hspace{50pt}%
        \begin{tabular}{ l | c | c | c | c }
            \textbf{Solver} & Euler & DDIM  & EDM   & DPM \\ \hline
            \textbf{FID}    & 10{,}96     & -     &   -   & -
        \end{tabular}
        \caption{Для 50 (слева) и 100 (справа) степов}
    \end{table}

    \subsection{Анализ результатов}
    
    имеет смысл написать про скорость методов (пока от самого быстрого к медленному:
    Euler --- EDM)


    \section{Разделение работы}
    \begin{itemize}
        \item Реализация солверов:
            \begin{itemize}
                \item Euler --- Фёдор;
                \item DDIM --- Фёдор;
                \item EDM --- Арсений;
                \item DPM --- Фёдор;
            \end{itemize}
        \item Составление архитектуры --- совместно;
        \item Реализация общего для солверов кода --- Арсений;
        \item Создание и оформление репозитория --- Арсений;
        \item Проведение экспериментов --- совместно;
        \item Составление отчёта --- совместно.
    \end{itemize}

\end{document}